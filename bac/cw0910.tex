\documentclass[12pt, fleqn]{extarticle}

\usepackage[russian]{babel}
\usepackage{amsmath, amssymb, amsfonts, amsthm}
\usepackage[utf8]{inputenc}

\usepackage{geometry}
\geometry{a4paper, top=2.0cm, left=2.0cm, right=2.0cm, bottom=2.0cm}

\usepackage{graphicx}
\usepackage[compact]{titlesec}
\usepackage{enumitem}
\usepackage{verbatim}
\usepackage{bm}

\setlength{\parskip}{0pt}
\setlength\parindent{0pt}

\newcommand{\nat}{\mathbb{N}}
\newcommand{\integer}{\mathbb{Z}}
\newcommand{\rational}{\mathbb{Q}}
\newcommand{\real}{\mathbb{R}}
\newcommand{\complex}{\mathbb{C}}


\DeclareMathOperator{\Kernel}{Ker}
\DeclareMathOperator{\Image}{Im}
\DeclareMathOperator{\characteristic}{char}



\newtheorem{theorem}{Теорема}
\newtheorem*{theorem_}{Теорема}
\newtheorem*{corollarb}{Следствие}
\newtheorem{proposition}{Предложение}
\newtheorem{claim}{Утверждение}
\newtheorem{lemma}{Лемма}

\theoremstyle{definition}
\newtheorem{definition}{Определение}
\newtheorem*{definition_}{Определение}
\newtheorem*{Not}{Обозначения}
\newtheorem{problem}{Задача}
\setcounter{problem}{0}

\theoremstyle{remark}
\newtheorem*{remark}{Замечание}


\begin{document}

\pagenumbering{gobble}
\clearpage
\thispagestyle{empty}
\subsubsection*{ВШЭ, 1 курс ПМИ. 10 сентября}

\subsubsection*{Алгоритм Евклида.}

\begin{problem}
Найдите $\text{НОД}(105369,4991)$.
\end{problem}

\begin{problem}
Докажите, что для каждого натурального $n$
числа $n! + 1$ и $(n + 1)! + 1$ взаимно просты.
\end{problem}

\begin{problem}
Докажите, что для любых $a, b \in \integer$ 
$\text{НОД}(a,b) = \text{НОД}(5a + 3b,13a + 8b)$.
\end{problem}


\begin{problem}
Найдите $\text{НОД}(3^{m} - 1, 3^{n} - 1)$
для $m, n \in \nat$.
\end{problem}

\begin{problem}
Определим последовательность чисел Фибоначчи:
$$
F_{n} =
\begin{cases}
0 & \text{при } n = 0 \\
1 & \text{при } n = 1\\
F_{n - 1} + F_{n - 2} & \text{if } n > 1\\
\end{cases}
$$
Чему равен $\text{НОД}(F_{n + 1}, F_{n})$?
\end{problem}

\begin{problem}
Изначально множество состоит из $n$ натуральных чисел.
На каждом шаге к нему можно добавить модуль разности любых двух элементов. 
Найдите:\\
а) минимальное число, которое можно добавить с помощью таких операций;\\
б) максимальное количество элементов, которые можно добавить с помощью таких операций.
\end{problem}

\begin{problem}
Для любого ли натурального $n$
существуют $a, b \in \integer$,
алгоритм Евклида для которых требует $n$ шагов?
\end{problem}

\begin{problem}
Найдите $\text{НОД}(62510,23731)$
и его линейное представление.
\end{problem}

\begin{problem}
Решите в целых числах уравнение
$$4439x + 1679y= 161$$
\end{problem}

\begin{problem}
Найти наименьшее натуральное $n$, 
при котором уравнение $8x + 13y = n$ 
будет иметь ровно 9 решений в натуральных числах.
\end{problem}


\begin{problem}
Пусть $a, b, c, n \in \integer$.
Найдите необходимое и достаточное условие того, что
уравнение
$$ax + by + cz = n$$
разрешимо в целых числах.
\end{problem}




\end{document}
