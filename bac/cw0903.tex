\documentclass[12pt, fleqn]{extarticle}

\usepackage[russian]{babel}
\usepackage{amsmath, amssymb, amsfonts, amsthm}
\usepackage[utf8]{inputenc}

\usepackage{geometry}
\geometry{a4paper, top=2.0cm, left=2.0cm, right=2.0cm, bottom=2.0cm}

\usepackage{graphicx}
\usepackage[compact]{titlesec}
\usepackage{enumitem}
\usepackage{verbatim}
\usepackage{bm}

\setlength{\parskip}{0pt}
\setlength\parindent{0pt}

\newcommand{\nat}{\mathbb{N}}
\newcommand{\integer}{\mathbb{Z}}
\newcommand{\rational}{\mathbb{Q}}
\newcommand{\real}{\mathbb{R}}
\newcommand{\complex}{\mathbb{C}}


\DeclareMathOperator{\Kernel}{Ker}
\DeclareMathOperator{\Image}{Im}
\DeclareMathOperator{\characteristic}{char}



\newtheorem{theorem}{Теорема}
\newtheorem*{theorem_}{Теорема}
\newtheorem*{corollarb}{Следствие}
\newtheorem{proposition}{Предложение}
\newtheorem{claim}{Утверждение}
\newtheorem{lemma}{Лемма}

\theoremstyle{definition}
\newtheorem{definition}{Определение}
\newtheorem*{definition_}{Определение}
\newtheorem*{Not}{Обозначения}
\newtheorem{problem}{Задача}
\setcounter{problem}{0}

\theoremstyle{remark}
\newtheorem*{remark}{Замечание}


\begin{document}

\pagenumbering{gobble}
\clearpage
\thispagestyle{empty}
\subsubsection*{ВШЭ, 1 курс ПМИ. 3 сентября}

\subsubsection*{Делимость. Деление с остатком.}

\begin{problem}
Докажите, что произведение любых пяти последовательных целых чисел
делится на 120.
\end{problem}

\begin{problem}
Сколько натуральных чисел, меньших 2000, не делятся ни на 12, ни на 14?
\end{problem}

\begin{problem}
Пусть $n$ -- составное число. 
Докажите, что существует простое $p|n$, такое что $p \leq n^{1/2}$.
\end{problem}

\begin{problem}
Можно ли числа от $1$ до $32$ разбить на несколько групп с одинаковыми произведениями?
\end{problem}

\begin{problem}
Докажите, что остаток от деления простого числа на 30 
равен либо $1$, либо простому числу.
\end{problem}

\begin{problem}
Докажите, что для любого натурального $n$ 
найдется $n$ подряд идущих составных чисел.
\end{problem}

\begin{problem}
Докажите признак делимости на 11: 
число, имеющее десятичную запись $a_n\dots a_0$ 
делится на 11 тогда и только тогда, когда $\sum\limits_{i=0}^n (-1)^ia_i$ 
делится на 11.
\end{problem}


\begin{problem}
Назовем натуральное число палиндромом, если оно читается одинаковым образом слева направо и справа налево. 
Найдите все палиндромы с четным количеством цифр, являющиеся простыми (т.е., делящиеся только на себя и 1).
\end{problem}


\begin{problem}
Докажите, что среди любых 18 последовательных трёхзначных чисел
найдется число, делящееся на сумму своих цифр.
\end{problem}


\begin{problem}
Пусть каждое из целых чисел $a$, $b$, $c$ и $d$ делится на $ab-cd$. 
Доказать, что тогда $ab-cd$ равно $1$ или $-1$.
\end{problem}

\begin{definition}
Пусть $a, b \in \integer$.
Разделить с остатком $a$ на $b$ -- это значит найти $q, r \in \integer$
такие, что $a = bq + r$ и $0 \leqslant r < |b|$.
\end{definition}

\begin{problem}
Доказать, что при любом целом $k > 1$ и любом $n \in \nat$
каждое число $a \in \{0, 1, \ldots k^{n} - 1\}$
можно единственным образом представить в виде
$$a = a_{0} + a_{1}k + \ldots + a_{n-1}k^{n-1},$$
где  $a_{i} \in \{0, 1, \ldots k - 1\}$. 
(Такое представление числа $a$ называют $k$-ичным.)
\end{problem}

\begin{problem}
Как определить, чётно ли число, по его записи в системе с
основанием $k$?
\end{problem}

\begin{problem}
Доказать, что при любом целом $k > 1$ и любом $n \in \nat$
каждое число $a \in \{0, 1, \ldots (n+1)! - 1\}$
можно единственным образом представить в виде
$$a = a_{1} \cdot 1! + a_{2} \cdot 2! + \ldots + a_{n} \cdot n!,$$
где  $a_{i} \in \{0, 1, \ldots i\}$. 
(Такое представление числа $a$ называют факториальным.)
\end{problem}


\begin{problem}
Докажите, что при любых натуральных $n$ и $m$ число 
$$\frac{1}{n} + \frac{1}{n + 1} + \ldots + \frac{1}{n + m}$$
не является целым.
\end{problem}



\end{document}
