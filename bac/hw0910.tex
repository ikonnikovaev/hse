\documentclass[12pt, fleqn]{extarticle}

\usepackage[russian]{babel}
\usepackage{amsmath, amssymb, amsfonts, amsthm}
\usepackage[utf8]{inputenc}

\usepackage{geometry}
\geometry{a4paper, top=2.0cm, left=2.0cm, right=2.0cm, bottom=2.0cm}

\usepackage{graphicx}
\usepackage[compact]{titlesec}
\usepackage{enumitem}
\usepackage{verbatim}
\usepackage{bm}

\setlength{\parskip}{0pt}
\setlength\parindent{0pt}

\newcommand{\nat}{\mathbb{N}}
\newcommand{\integer}{\mathbb{Z}}
\newcommand{\rational}{\mathbb{Q}}
\newcommand{\real}{\mathbb{R}}
\newcommand{\complex}{\mathbb{C}}


\DeclareMathOperator{\Kernel}{Ker}
\DeclareMathOperator{\Image}{Im}
\DeclareMathOperator{\characteristic}{char}



\newtheorem{theorem}{Теорема}
\newtheorem*{theorem_}{Теорема}
\newtheorem*{corollarb}{Следствие}
\newtheorem{proposition}{Предложение}
\newtheorem{claim}{Утверждение}
\newtheorem{lemma}{Лемма}

\theoremstyle{definition}
\newtheorem{definition}{Определение}
\newtheorem*{definition_}{Определение}
\newtheorem*{Not}{Обозначения}
\newtheorem{problem}{Задача}
\setcounter{problem}{0}

\theoremstyle{remark}
\newtheorem*{remark}{Замечание}


\begin{document}

\pagenumbering{gobble}
\clearpage
\thispagestyle{empty}
\subsubsection*{ВШЭ, 1 курс ПМИ. Домашнее задание от 10 сентября}

\subsubsection*{Алгоритм Евклида.}

\begin{problem}[1 балл]
Найдите $\text{НОД}(7^{13} - 2,7^{11} + 2)$.
\end{problem}

\begin{problem}[1 балл]
Какие целые значения может принимать
$\text{НОД}(7a^{2} + 2a + 1,3a + 1)$
при различных целых $a$?
\end{problem}


\begin{problem}[1 балл]
Определим последовательность чисел Фибоначчи:
$$
F_{n} =
\begin{cases}
0 & \text{при } n = 0 \\
1 & \text{при } n = 1\\
F_{n - 1} + F_{n - 2} & \text{if } n > 1\\
\end{cases}
$$
Чему равен $\text{НОД}(F_{n + 1}, F_{n})$?
\end{problem}

\begin{problem}
Изначально множество состоит из $n$ натуральных чисел.
На каждом шаге к нему можно добавить модуль разности любых двух элементов. 
Найдите:\\
а) (1 балл) минимальное число, которое можно добавить с помощью таких операций;\\
б) (1 балл) максимальное количество элементов, которые можно добавить с помощью таких операций.
\end{problem}

\begin{problem}[1 балл]
Для любого ли натурального $n$
существуют $a, b \in \integer$,
алгоритм Евклида для которых требует $n$ шагов?
\end{problem}




\end{document}
