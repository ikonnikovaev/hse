\documentclass[12pt, fleqn]{extarticle}

\usepackage[russian]{babel}
\usepackage{amsmath, amssymb, amsfonts, amsthm}
\usepackage[utf8]{inputenc}

\usepackage{geometry}
\geometry{a4paper, top=2.0cm, left=2.0cm, right=2.0cm, bottom=2.0cm}

\usepackage{graphicx}
\usepackage[compact]{titlesec}
\usepackage{enumitem}
\usepackage{mdwlist}
\usepackage{verbatim}

\usepackage{tikz}
\usepackage{hyperref}


\setlength{\parskip}{0pt}
\setlength\parindent{0pt}

\newcommand{\nat}{\mathbb{N}}
\newcommand{\integer}{\mathbb{Z}}
\newcommand{\rational}{\mathbb{Q}}
\newcommand{\real}{\mathbb{R}}
\newcommand{\complex}{\mathbb{C}}
\newcommand{\characteristic}{\textrm{char }}
\newcommand{\Legendre}[2]{\left(\frac{#1}{#2}\right)}
\newcommand{\anglebrackets}[1]{\left\langle#1\right\rangle}

\DeclareMathOperator{\Kernel}{Ker}
\DeclareMathOperator{\Image}{Im}
\DeclareMathOperator{\Characteristic}{Im}
\DeclareMathAlphabet{\mathbbold}{U}{bbold}{m}{n}


\newtheorem{theorem}{Теорема}
\newtheorem*{theorem_}{Теорема}
\newtheorem*{corollary}{Следствие}
\newtheorem{proposition}{Предложение}
\newtheorem*{claim}{Утверждение}
\newtheorem{lemma}{Лемма}

\theoremstyle{definition}
\newtheorem{definition}{Определение}
\newtheorem*{definition_}{Определение}
\newtheorem*{pseudodefinition}{Псевдоопределение}
\newtheorem*{Not}{Обозначение}
\newtheorem{problem}{Задача}
\newtheorem*{example}{Пример}


\theoremstyle{remark}
\newtheorem*{remark}{Замечание}

\begin{document}
\pagenumbering{gobble}
\clearpage

\thispagestyle{empty}
\subsubsection*{Магистратура ВШЭ. Домашнее задание от 4 сентября.}


\begin{definition}
Пусть $V$ -- векторное пространство над полем $K$. 
{\it Линейная комбинация} векторов (т.е., элементов $V$) $v_{1}, v_{2}, \ldots, v_{n}$ 
-- это любое выражение вида 
$$\alpha_{1}v_{1} + \alpha_{2}v_{2} + \ldots + \alpha_{n}v_{n}, \quad \alpha_{i} \in K$$

$v_{1}, v_{2}, \ldots, v_{n}$ называется линейно зависимыми, если
существует их нетривиальная (т.е., такая, что не все $\alpha_{i}$ равны 0)
линейная комбинация, равная $0 \in V$. Если  такой линейной комбинации нет,
$v_{1}, v_{2}, \ldots, v_{n}$ называется линейно независимыми.

Множество всех линейных комбинаций $v_{1}, v_{2}, \ldots, v_{n}$ 
называется их {\it линейной оболочкой} и обозначается
$\langle v_{1}, v_{2}, \ldots, v_{n} \rangle$.
\end{definition}

 
\begin{definition}
$\{v_{1}, v_{2}, \ldots, v_{n}\}$ -- множество образующих $V$,
если $\langle v_{1}, v_{2}, \ldots, v_{n} \rangle = V$.

Линейно независимое множество образующих называется {\it базисом}.
Все базисы векторного пространства $V$ имеют равное число элементов, которое называется {\it размерностью} пространства.

Т.е., пусть $V$ конечномерно, $(v_{1}, v_{2}, \ldots, v_{n})$ -- его базис.
Тогда любой вектор $w \in V$ представляется в  виде 
$$w = \lambda_{1}v_{1} + \lambda_{2}v_{2} + \ldots + \lambda_{n}v_{n}$$
$(\lambda_{1}, \lambda_{2}, \ldots, \lambda_{n})$ называются координатами вектора $w$
относительно базиса $(v_{1}, v_{2}, \ldots, v_{n})$.
\end{definition}


\begin{problem}[1 балл]
Найти базис пространства $\real^{3}$ , в котором векторы $x$, $y$, $z$
имеют координатные столбцы $[x]$, $[y]$, $[z]$.
$$x = (9, 2, 0), \quad y = (6, 3, 4), \quad z = (3, 1, 2),$$
$$[x] = (1, 2, 1)^{T}, \quad [y] = (1, -1, 2)^{T}, \quad [z] = (-2, -1, 3)^{T}$$
\end{problem}

\begin{definition}
{\it Прямое произведение} $U \times V$ векторных пространств $U$ и $V$ над полем $K$-- это множество пар
$(u, v)$, где $u \in U$, $v \in V$, с операциями\\
$(u_1, v_1) + (u_2, v_2) = (u_1 + u_2, v_1 + v_2)$;\\
$\alpha (u, v) = (\alpha u, \alpha v)$,\\
которые превращают его также в векторное пространство над $K$.

Аналогично определяется прямое произведение $n$ векторных пространств $V_1, \ldots V_n$.
\end{definition}

\begin{problem}[1 балл]
Докажите, что размерность $U \times V$ равна $m + n$. 
\end{problem}

\begin{definition}
$U \subseteq V$ называется подпространством $V$ (обозначение: $U \leqslant V$), если\\
1) $u_{1} + u_{2} \in U$ для любых $u_{1}, u_{2} \in U$;\\
2) $\lambda u \in U$ для любых $u \in U, \lambda \in K$;
\end{definition}


\begin{problem}[1 балл]
Пусть $V_1,V_2\leq V$. 
Докажите, что если $V_1\cup V_2\leq V$, то $V_1\leq V_2$ или $V_2\leq V_1$.
\end{problem}


\begin{problem}
Найдите размерность пространства:\\
а) (1 балл) кососимметричных матриц (т. е., таких, что $A = A^{T}$) размера $n\times n$;\\
б) (1 балл) матриц размера $n\times n$, коммутирующих с $e_{12}$ 
($e_{ij}$ -- матрица с единицей на позиции $(i, j)$ и нулями на остальных),
т.е., таких, что $Ae_{12} = e_{12}A$
\end{problem}

\begin{problem}
Пусть $V=\{f\in K[x]\,|\, \deg f\leq n\} = \{a_{0} + a_{1}t + \ldots + a_{n}t^{n}, a_{i} \in K\}$
(многочлены с коэффициентами из поля  $k$ степени не выше $n$). \\
а) (1 балл) Покажите, что любой набор многочленов $p_0(x), p_1(x), \dots, p_n(x)$, 
такой, что $\deg p_i(x)=i$, является базисом $V$.\\
б) (2 балла) Пусть даны различные элементы $\lambda_0, \dots, \lambda_n\in K$. 
Покажите, что набор многочленов $p_i(x)=\prod_{j\neq i}(x-\lambda_j)$ является базисом $V$.
\end{problem}


\begin{problem}[1 балл]
Пусть 
$$A=\left(\begin{matrix}
3& 5&-4& 2 \\
2& 4&-6& 3 \\
11& 17& -8& 4 
\end{matrix}\right)$$
Найдите базис пространства решений однородного уравнения $Ax = 0$.
\end{problem}

\begin{problem}
а) [1 балл] Покажите, что множество решений уравнения $a_1x_1+a_2x_2+\cdots+a_nx_n=0$, 
$a_{i} \in K$, имеет размерность $n-1$ над $K$. \\
б) [2 балла] Покажите, что все подпространства размерности $n-1$ в $K^n$ имеют такой вид.
\end{problem}

\end{document}
