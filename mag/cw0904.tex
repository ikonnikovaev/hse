\documentclass[12pt, fleqn]{extarticle}

\usepackage[russian]{babel}
\usepackage{amsmath, amssymb, amsfonts, amsthm}
\usepackage[utf8]{inputenc}

\usepackage{geometry}
\geometry{a4paper, top=2.0cm, left=2.0cm, right=2.0cm, bottom=2.0cm}

\usepackage{graphicx}
\usepackage[compact]{titlesec}
\usepackage{enumitem}
\usepackage{mdwlist}
\usepackage{verbatim}

\usepackage{tikz}
\usepackage{hyperref}


\setlength{\parskip}{0pt}
\setlength\parindent{0pt}

\newcommand{\nat}{\mathbb{N}}
\newcommand{\integer}{\mathbb{Z}}
\newcommand{\rational}{\mathbb{Q}}
\newcommand{\real}{\mathbb{R}}
\newcommand{\complex}{\mathbb{C}}
\newcommand{\characteristic}{\textrm{char }}
\newcommand{\Legendre}[2]{\left(\frac{#1}{#2}\right)}
\newcommand{\anglebrackets}[1]{\left\langle#1\right\rangle}

\DeclareMathOperator{\Kernel}{Ker}
\DeclareMathOperator{\Image}{Im}
\DeclareMathOperator{\Characteristic}{Im}
\DeclareMathAlphabet{\mathbbold}{U}{bbold}{m}{n}


\newtheorem{theorem}{Теорема}
\newtheorem*{theorem_}{Теорема}
\newtheorem*{corollary}{Следствие}
\newtheorem{proposition}{Предложение}
\newtheorem*{claim}{Утверждение}
\newtheorem{lemma}{Лемма}

\theoremstyle{definition}
\newtheorem{definition}{Определение}
\newtheorem*{definition_}{Определение}
\newtheorem*{pseudodefinition}{Псевдоопределение}
\newtheorem*{Not}{Обозначение}
\newtheorem{problem}{Задача}
\newtheorem*{example}{Пример}


\theoremstyle{remark}
\newtheorem*{remark}{Замечание}

\begin{document}
\pagenumbering{gobble}
\clearpage

\thispagestyle{empty}
\subsubsection*{Магистратура ВШЭ. 4 сентября.}

\begin{definition}
Пусть $K$ — поле. Множество $V$ с операциями сложения
и умножения на элемент $K$ называется векторным пространством над $K$, если
для любых $u, v, w \in V$, $\alpha, \beta \in K$
\begin{enumerate}[topsep = 0ex, itemsep = -0.9ex]
\item $(u + v) + w = u + (v + w)$ для любых $u, v, w \in V$;
\item существует $0 \in V$, такой что $0 + v = v + 0 = v$ для любого $v \in V$;
\item для любого $v \in V$ найдется $-v \in V$, такой что $v + (-v) = (-v) + v = 0$;
\item $u + v = v + u$ для любых $u, v \in V$;
\item $\alpha (u + v) = \alpha u + \alpha v$ для любых $u, v \in V, \alpha \in K$;
\item $(\alpha + \beta)v = \alpha v + \beta v$ для любых $v \in V, \alpha, \beta \in K$;
\item $(\alpha\beta)v = \alpha(\beta v)$ для любых $v \in V, \alpha, \beta \in K$;
\item $1 \cdot v = v$ для любого $v \in V$;
\end{enumerate}

\end{definition}

\begin{example}
$\real^{n}$ -- множество наборов из $n$ вещественных чисел -- векторное пространство над $\real$.
Его можно (и часто удобно) отождествить с множеством матриц размера $n \times 1$ (столбцов)
и записывать вместо $(a_{1}, \ldots, a_{n})$
$$\left( \begin{array}{c}
 a_{1}\\
\vdots \\
a_{n} \end{array} \right)
$$
\end{example}


\begin{definition}
Пусть $V$ -- векторное пространство над полем $K$. 
{\it Линейная комбинация} векторов (т.е., элементов $V$) $v_{1}, v_{2}, \ldots, v_{n}$ 
-- это любое выражение вида 
$$\alpha_{1}v_{1} + \alpha_{2}v_{2} + \ldots + \alpha_{n}v_{n}, \quad \alpha_{i} \in K$$

$v_{1}, v_{2}, \ldots, v_{n}$ называется линейно зависимыми, если
существует их нетривиальная (т.е., такая, что не все $\alpha_{i}$ равны 0)
линейная комбинация, равная $0 \in V$. Если  такой линейной комбинации нет,
$v_{1}, v_{2}, \ldots, v_{n}$ называется линейно независимыми.

Множество всех линейных комбинаций $v_{1}, v_{2}, \ldots, v_{n}$ 
называется их {\it линейной оболочкой} и обозначается
$\langle v_{1}, v_{2}, \ldots, v_{n} \rangle$.
\end{definition}

\begin{problem}
Поле $\real$ можно рассмотреть как векторное пространство над $\rational$.
Доказать, что в нем $1, \sqrt{2}, \sqrt{3}$ линейно независимы.
\end{problem}

\begin{problem}
Являются ли линейно независимыми в $\real^{3}$\\
а) $(1, -1, 2), \quad(-1, 0, 3), \quad(-4, -3, 27)$;\\
б) $(2, 1, -3), \quad(3, 2, -5), \quad(1, -1, 1)$?
\end{problem}

\begin{problem}
Пусть $K = \real$, $V = \real^{3}$ -- т.е., тройки вещественных чисел, операции определяются обычным образом:\\
$(x_1, y_1, z_1) + (x_2, y_2, z_2) = (x_1 + x_2, y_1 + y_2, z_1 + z_2)$,\\
$\alpha(x, y, z) = (\alpha x, \alpha y, \alpha z)$.
 
Найти общий вид вектора из пересечения
$\langle v_{1}, v_{2}\rangle$ и $\langle u_{1}, u_{2}\rangle$,
если $v_{1} = (0, 2, -1)$, $v_{2} = (1, -1, 1)$, 
$u_{1} = (1, 1, 2)$, $u_{2} = (1, 0, -1)$.
\end{problem}

\begin{definition}
$\{v_{1}, v_{2}, \ldots, v_{n}\}$ -- множество образующих $V$,
если $\langle v_{1}, v_{2}, \ldots, v_{n} \rangle = V$.
(Можно говорить и о бесконечных множествах образующих, но их мы обсуждать не будем.)

Линейно независимое множество образующих называется {\it базисом}.
Все базисы векторного пространства $V$ имеют равное число элементов, которое называется {\it размерностью} пространства.

Т.е., пусть $V$ конечномерно, $(v_{1}, v_{2}, \ldots, v_{n})$ -- его базис.
Тогда любой вектор $w \in V$ представляется в  виде 
$$w = \lambda_{1}v_{1} + \lambda_{2}v_{2} + \ldots + \lambda_{n}v_{n}$$
$(\lambda_{1}, \lambda_{2}, \ldots, \lambda_{n})$ называются координатами вектора $w$
относительно базиса $(v_{1}, v_{2}, \ldots, v_{n})$.
\end{definition}

\begin{remark}
Здесь мы говорим о базисе как об упорядоченном наборе, а не просто множестве векторов
(т.е., нам важна нумерация элементов). Чтобы подчеркнуть это, мы 
перечисляем элементы базиса в круглых скобках, а не в фигурных.
\end{remark}


\begin{problem}
Дополнить пару векторов $(1, 1, 0, 0)$ и $(1, 1, 1, 1)$  
до базиса пространства $\mathbb{F}_2^{4}$.
\end{problem}


\begin{problem}
В пространстве $M(2, \real)$ (матрицы $2 \times 2$ c элементами из $\real$)
укажите какой-нибудь базис, содержащий матрицы 
$\left( \begin{array}{cc}
1 & 1\\
1 & 1 \end{array} \right)$ и
$\left( \begin{array}{cc}
0 & 1\\
-1 & 0 \end{array} \right)$

Найдите координаты единичной матрицы в выбранном базисе.
\end{problem}

\begin{problem}
Найти базис пространства $\real^{3}$ , в котором векторы $x$, $y$, $z$
имеют координатные столбцы $[x]$, $[y]$, $[z]$.
$$x = (0, -1, -1), \quad y = (2, -4, 3), \quad z = (6, -6, 5),$$
$$[x] = (1, 1, 1)^{T}, \quad [y] = (2, 3, 2)^{T}, \quad [z] = (5, 6, 4)^{T}$$
\end{problem}

\begin{definition}
{\it Прямое произведение} $U \times V$ векторных пространств $U$ и $V$ над полем $K$-- это множество пар
$(u, v)$, где $u \in U$, $v \in V$, с операциями\\
$(u_1, v_1) + (u_2, v_2) = (u_1 + u_2, v_1 + v_2)$;\\
$\alpha (u, v) = (\alpha u, \alpha v)$,\\
которые превращают его также в векторное пространство над $K$.

Аналогично определяется прямое произведение $n$ векторных пространств $V_1, \ldots V_n$.
\end{definition}

\begin{problem}
Докажите, что размерность $U \times V$ равна $m + n$. 
\end{problem}

\begin{definition}
$U \subseteq V$ называется подпространством $V$ (обозначение: $U \leqslant V$), если\\
1) $u_{1} + u_{2} \in U$ для любых $u_{1}, u_{2} \in U$;\\
2) $\lambda u \in U$ для любых $u \in U, \lambda \in K$;
\end{definition}


\begin{problem}
Пусть $V_1,V_2\leq V$. 
Докажите, что если $V_1\cup V_2\leq V$, то $V_1\leq V_2$ или $V_2\leq V_1$.
\end{problem}


\begin{problem}
Выяснить, является ли подмножество $U$ пространства $V = K[t]$
его подпространством, и в случае положительного ответа найти
какой-нибудь его базис:\\
а) $U = \left\{f \vert f''' = 0 \right\}$;\\
б) $U = \left\{f \vert f(1) = 0 \right\}$;\\
в) $U = \left\{f \vert f(0) = 1 \right\}$;\\
г) $U = \left\{f \vert f(0) + f(1) = 0 \right\}$.
\end{problem}


\begin{problem}
Найдите размерность пространства:\\
а) кососимметричных матриц (т. е., таких, что $A = A^{T}$) размера $n\times n$;\\
б) $K[x_1,\dots,x_n]_{\leq k}=\{ f \in K[x_1,\dots,x_n]\,|\, \deg f\leq k\}$;\\
в) матриц, коммутирующих с $e_{12}$ 
($e_{ij}$ -- матрица с единицей на позиции $(i, j)$ и нулями на остальных).
\end{problem}

\begin{problem}
Пусть $V=\{f\in K[x]\,|\, \deg f\leq n\}$. \\
а)Покажите, что любой набор многочленов $p_0(x), p_1(x), \dots, p_n(x)$, 
такой, что $\deg p_i(x)=i$, является базисом $V$.\\
б) Пусть даны различные элементы $\lambda_0, \dots, \lambda_n\in K$. 
Покажите, что набор многочленов $p_i(x)=\prod_{j\neq i}(x-\lambda_j)$ является базисом $V$.
\end{problem}


\begin{problem}
Пусть 
$$A=\left(\begin{matrix}
3& 5&-4& 2 \\
2& 4&-6& 3 \\
11& 17& -8& 4 
\end{matrix}\right)$$
Найдите базис пространства решений однородного уравнения $Ax=0$.
\end{problem}

\begin{problem}
Покажите, что множество решений уравнения $a_1x_1+a_2x_2+\cdots+a_nx_n=0$, имеет размерность $n-1$ над $K$. Покажите, что все подпространства размерности $n-1$ в $K^n$ имеют такой вид.
\end{problem}


\begin{problem}
Найти базис пересечения и суммы подпространств 
$U=\langle u_1,u_2,u_3\rangle$, $V=\langle v_1,v_2,v_3\rangle$ в $\real^4$, если
$$
u_1=\left(\begin{matrix}
1 \\2\\ 0\\ 1 
\end{matrix}\right),\,\,
u_2=\left(\begin{matrix}
1\\ 1\\ 1\\ 0
\end{matrix}\right),\,\,
u_3=\left(\begin{matrix}
0\\ 1\\ -1\\ 1 
\end{matrix}\right),\,\,
v_1=\left(\begin{matrix}
1\\ 0\\ 1\\  0
\end{matrix}\right),\,\,
v_2=\left(\begin{matrix}
1 \\3 \\0 \\1
\end{matrix}\right),\,\,
v_3=\left(\begin{matrix}
0 \\3 \\-1 \\1
\end{matrix}\right).
$$
\end{problem}

\begin{problem}
Пусть $v_{1}, v_{2}, \ldots, v_{n} \in V$, 
$$D(v_{1}, v_{2}, \ldots, v_{n}) = 
\left\{(\lambda_{1}, \ldots \lambda_{n})\in K^{n} \vert
\lambda_{1}v_{1} + \ldots  + \lambda_{n}v_{n} = 0\right\}$$ 
Доказать, что:\\
а) $D(v_{1}, v_{2}, \ldots, v_{n}) \leqslant K^{n}$;\\
б) $D(v_{1}, v_{2}, \ldots, v_{n}) = K^{n}$ тогда и только тогда, когда 
$v_{1} = \ldots = v_{n} = 0$;\\
в) $\dim D(v_{1}, v_{2}, \ldots, v_{n}) + 
\dim\left(\left\langle v_{1}, v_{2}, \ldots, v_{n} \right\rangle\right) = n$.

\end{problem}




\begin{problem}
Пусть $V_1$, $V_2$, $V_3$ -- подпространства $W$. Покажите, что число $\dim (V_i + V_j)\cap V_k +\dim V_i\cap V_j$ одинаково для любой $(i,j,k)$ --- перестановки на множестве $\{1,2,3\}$.
\end{problem}

\begin{problem}
Пусть $V = \mathbb{F}_2^{n}$, $U \leqslant V$, $\dim(U) = m$
(в теории информации такое $U$ называется двоичным линейным $(n, m)$-кодом, а его
элементы -- кодовыми словами). Доказать, что в двоичном линейном коде:\\
а) (1 балл) либо все кодовые слова имеют четный вес Хэмминга, либо
ровно половина кодовых слов имеет четный вес, а вторая
половина -- нечетный (весом Хэмминга называется число ненулевых компонент вектора);\\
б) (1 балл) либо все кодовые слова начинаются с 0, либо ровно половина
кодовых слов начинается с 0, а вторая половина -- с 1.
\end{problem}



\end{document}
