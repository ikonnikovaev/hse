\documentclass[12pt, fleqn]{extarticle}

\usepackage[russian]{babel}
\usepackage{amsmath, amssymb, amsfonts, amsthm}
\usepackage[utf8]{inputenc}

\usepackage{geometry}
\geometry{a4paper, top=2.0cm, left=2.0cm, right=2.0cm, bottom=2.0cm}

\usepackage{graphicx}
\usepackage[compact]{titlesec}
\usepackage{enumitem}
\usepackage{mdwlist}
\usepackage{verbatim}

\usepackage{tikz}
\usepackage{hyperref}


\setlength{\parskip}{0pt}
\setlength\parindent{0pt}

\newcommand{\nat}{\mathbb{N}}
\newcommand{\integer}{\mathbb{Z}}
\newcommand{\rational}{\mathbb{Q}}
\newcommand{\real}{\mathbb{R}}
\newcommand{\complex}{\mathbb{C}}
\newcommand{\Legendre}[2]{\left(\frac{#1}{#2}\right)}
\newcommand{\curvedbrackets}[1]{\left\{#1\right\}}
\newcommand{\anglebrackets}[1]{\left\langle#1\right\rangle}

\DeclareMathOperator{\Kernel}{Ker}
\DeclareMathOperator{\Image}{Im}
\DeclareMathOperator{\Characteristic}{char}
\DeclareMathOperator{\Hom}{Hom}
\DeclareMathOperator{\End}{End}

\newtheorem{theorem}{Теорема}
\newtheorem*{theorem_}{Теорема}
\newtheorem*{corollary}{Следствие}
\newtheorem{proposition}{Предложение}
\newtheorem*{claim}{Утверждение}
\newtheorem{lemma}{Лемма}

\theoremstyle{definition}
\newtheorem{definition}{Определение}
\newtheorem*{definition_}{Определение}
\newtheorem*{pseudodefinition}{Псевдоопределение}
\newtheorem*{Not}{Обозначение}
\newtheorem{problem}{Задача}
\newtheorem*{example}{Пример}


\theoremstyle{remark}
\newtheorem*{remark}{Замечание}

\begin{document}
\pagenumbering{gobble}
\clearpage

\thispagestyle{empty}
\subsubsection*{Магистратура ВШЭ. 11 сентября.}

\begin{problem}
Найти базис пересечения и суммы подпространств 
$U=\langle u_1,u_2,u_3\rangle$, $V=\langle v_1,v_2,v_3\rangle$ в $\real^4$, если
$$
u_1=\left(\begin{matrix}
1 \\2\\ 0\\ 1 
\end{matrix}\right),\,\,
u_2=\left(\begin{matrix}
1\\ 1\\ 1\\ 0
\end{matrix}\right),\,\,
u_3=\left(\begin{matrix}
0\\ 1\\ -1\\ 1 
\end{matrix}\right),\,\,
v_1=\left(\begin{matrix}
1\\ 0\\ 1\\  0
\end{matrix}\right),\,\,
v_2=\left(\begin{matrix}
1 \\3 \\0 \\1
\end{matrix}\right),\,\,
v_3=\left(\begin{matrix}
0 \\3 \\-1 \\1
\end{matrix}\right).
$$
\end{problem}

\begin{definition}
Отображение $L: V \rightarrow W$ называется линейным отображением
(или гомоморфизмом векторных пространств), если для любых $v_{1}, v_{2}, v \in V$,
$\lambda \in K$\\
$L(v_{1} + v_{2}) = L(v_{1}) + L(v_{2})$ и $L(\lambda v) = \lambda L(v)$.

Множество линейных отображений из $V$ в $W$ будем обозначать $\Hom(V, W)$. 
Введем также обозначение $\End(V) = \Hom(V, V)$.
\end{definition}

\begin{problem}
Будет ли отображение $L \colon \complex \to \complex$, 
где $L(x+iy)=x-iy$, $\real$-линейным? $\complex$-линейным?
\end{problem}

\begin{problem}
Доказать, что произвольное линейное отображение 
переводит любое линейно зависимое семейство векторов в линейно зависимое. 
Верен ли аналогичный факт для линейно независимых семейств?
\end{problem}


\begin{problem} 
Пусть $L \in \Hom(V, W)$, $V_{1}, V_{2} \leqslant V$,
$W_{1}, W_{2} \leqslant W$.
Верны ли перечисленные ниже равенства? 
Если нет, то замените их на верные включения.

а) $L(V_{1} + V_{2}) = L(V_{1}) + L(V_{2})$\\
б) $L(V_{1} \cap V_{2}) = L(V_{1}) \cap L(V_{2})$\\
в) $L^{-1}(W_{1} + W_{2}) = 
L^{-1}(W_{1}) + L^{-1}(W_{2})$\\
г) $L^{-1}(W_{1} \cap W_{2}) =
 L^{-1}(W_{1}) \cap L^{-1}(W_{2})$
 
(Замечание: отображение $L$ не обязательно обратимо! 
$L^{-1}(W_{i}) = \curvedbrackets{v \in V \vert L(v) \in W_{i}}$ 
-- полный прообраз $W_{i}$.)
\end{problem}

\begin{problem} 
Покажите, что $\End(V)$ -- векторное пространство над $K$.
Докажите, что если образы двух ненулевых операторов из
$\End(V)$ различны, то эти операторы линейно независимы.
\end{problem}


\begin{definition}
Пусть $(v_{1}, \ldots, v_{n})$ -- базис $V$, 
$(w_{1}, \ldots, w_{m})$ -- базис $W$
Матрица $A = (a_{ij})$ называется матрицей линейного отображения 
$L: V \rightarrow W$, если 
$$L(v_{j}) = \sum\limits_{i = 1}^{m} a_{ij}w_{i}, \quad j = 1, \ldots n$$
\end{definition}


\begin{problem}
Пусть $V=\{ f\in K[x,y]\,|\, \deg f\leq 2\}$. 
Выпишите матрицу линейного отображения $L\colon V\to V$ такого, что 
$$L(f)=f(x,x+y)-f(x+y,-x+1)$$  
в мономиальном базисе. Найдите базис ядра и образа этого линейного отображения.
\end{problem}

\begin{problem} 
Пусть $K = \real$, 
$V = \real^{3}$, $W = \curvedbrackets{f \in \real[t] \vert f''' = 0}$.
Линейно ли отображение $L: V \rightarrow W$:
$$L((\alpha, \beta, \gamma)) =
 \alpha(t - 1)^{2} + \beta(t^{2} - 1) + \gamma(t^{2} - 3t + 2)$$

Если да, найдите его матрицу относительно какой-нибудь
пары базисов $V$ и $W$, а также укажите базисы ядра и образа $L$.
\end{problem}

\begin{definition}
Матрица перехода от базиса $(u_{1}, u_{2}, \ldots, u_{n})$
к базису $(v_{1}, v_{2}, \ldots, v_{n})$ -- это матрица $C$,
такая что
$$v_{j} = \sum\limits_{i = 1}^{n}c_{ij}u_{i}$$
\end{definition}

\begin{example}
Матрица перехода от базиса  $$((1, 0, 0), (0, 1, 0), (0, 0, 1))$$
к базису
$$((2, 1, 0), (-1, -2, -3), (-1, 0, 0))$$
в $\real^{3}$  -- это
$$\left( \begin{array}{ccc}
2 & -1 & -1\\
1 & -2 & 0\\
0 & -3 & 0 \end{array} \right)$$
(проверьте!)
\end{example}

\begin{remark}
Перечисляя базисные векторы, мы пишем их в круглых скобках,
как упорядоченный набор (а не в фигурных, как обычно записываем множества),
так как, как только мы начинаем говорить о координатах 
в этом базисе, нам становится важна нумерация.
(Подумайте, как изменится матрица перехода от $(u_{1}, u_{2}, \ldots, u_{n})$
к $(v_{1}, v_{2}, \ldots, v_{n})$, если перенумеровать $u_i$? Если перенумеровать $v_j$?)
\end{remark}

\begin{claim}
Если $C$ -- матрица перехода от базиса $\mathcal{B}$ к базису $\mathcal{B'}$, 
то $C$ обратима и $C^{-1}$ -- матрица перехода от $\mathcal{B'}$ к $\mathcal{B}$. 
\end{claim}

\begin{problem}
В пространстве $K[t]_3$ (многочленов с коэффициентами из поля $K$, степени не более 3)
найти матрицу перехода от базиса $\mathcal{B}$
к базису $\mathcal{B'}$:
$$\mathcal{B} = (1, 1 + t, (1 + t)^{2} , (1 + t)^{3}),$$
$$\mathcal{B'} = (t^{3} , t^{3} - t^{2} , t^{3} - t, t^{3} - 1)$$
\end{problem}


\begin{problem}
В пространстве $\real^3$ 
рассмотрим базис $f_1 = (1, 0, 1)$, $f_2=(1, 1, 1)$, $f_3=(2, 1, 1)$. \\
а) Найдите матрицу перехода из стандартного базиса в базис $f$.\\
б) Найдите матрицу линейного отображения $L\colon \real^3\to \real^3$ такого, что 
$$L(x_i)=y_i,$$ 
где 
$$x_1 = (1, 2, 1), \quad x_2 = (1, 1, 1), \quad x_3 = (0, 1, 1),$$
$$y_1=(-3, -1, -1), \quad y_2 = (2, 1, 2) \quad y_3 = (3, 1, 3),$$
в базисе $f$.
\end{problem}

\begin{problem}
Пусть $\dim(V_{1}) + \dim(W_{1}) = \dim(V)$ для некоторых 
подпространств $V_{1} \leqslant V$, $W_{1} \leqslant W$.
Докажите, что существует отображение
$L \in \Hom(V, W)$ такое, что 
$\Image(L) = W_{1}$, $\Kernel(L) = V_{1}$.
\end{problem}



\end{document}
